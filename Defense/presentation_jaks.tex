\documentclass[czech]{beamer}

\usetheme{Rochester}

\usepackage{amsmath}
\usepackage{caption}
\usepackage{subcaption}
\usepackage{babel}

\title{Robustní strojové učení a adversariální vzorky}
\author{Pavel Jakš}
\institute{Matematická informatika, FJFI ČVUT v Praze}

\begin{document}

\selectlanguage{czech}

\frame{\titlepage}

\begin{frame}
    \frametitle{Obsah}
    \tableofcontents
\end{frame}

\section{Prostředí}

\subsection{Neuronové sítě}

% \begin{frame}
%     \frametitle{Neuronová síť}
%     \begin{itemize}
%         % \item Výpočetní model
%         % \item Sestává z tzv. vrstev \cite{Goodfellow}
%         % \item Vhodná pro \emph{regresi} či \emph{klasifikaci}
%         \item Odpovídá jí zobrazení $F_\theta: \mathbb{R}^{n_1 \times ... \times n_k} \rightarrow \mathbb{R}^{m_1 \times ... \times m_l}$
%         \begin{itemize}
%             \item $\theta$ jsou \emph{parametry neuronové sítě}
%             \item Jedná se o zobrazení složené z tzv. vrstev \cite{Goodfellow}
%         \end{itemize}
%     \end{itemize}
% \end{frame}

% \subsection{Učení neuronové sítě}

\begin{frame}
    \frametitle{Neuronová síť}
    \begin{itemize}
        % \item Výpočetní model
        % \item Sestává z tzv. vrstev \cite{Goodfellow}
        % \item Vhodná pro \emph{regresi} či \emph{klasifikaci}
        \item Odpovídá jí zobrazení $F_\theta: \mathbb{R}^{n_1 \times ... \times n_k} \rightarrow \mathbb{R}^{m_1 \times ... \times m_l}$
        \begin{itemize}
            \item $\theta$ jsou \emph{parametry neuronové sítě}
            \item Jedná se o zobrazení složené z tzv. vrstev \cite{Goodfellow}
        \end{itemize}
        \item Hledání vhodných parametrů $\theta$ pro neuronovou síť
        \begin{itemize}
            \item Převedení na optimalizaci vhodné ztrátové funkce
        \end{itemize}
        % \item Volba vhodné \emph{účelové funkce}
        % \item Následná \emph{optimalizace} tohoto kritéria
    \end{itemize}
    \begin{block}{Častá volba sestává z dílčích ztrát}
        \centering
        $J(\theta) = \frac{1}{N} \sum_{i=1}^N L\left(F_\theta(x^{(i)}), y^{(i)}\right)$
    \end{block}
    \begin{itemize}
        \item Tento přístup vyžaduje existenci \emph{trénovací datové sady}
        \begin{itemize}
            \item Uspořádaná dvojice 
            $\mathbb{T} = \left(\left\{x^{(i)} | i \in \{1, ..., N\}\right\}, \left\{y^{(i)} | i \in \{1, ..., N\}\right\}\right)$
            % \item $\mathbb{X} = \left\{x^{(i)} | i \in \{1, ..., N\}\right\}$
            % \item $\mathbb{Y} = \left\{y^{(i)} | i \in \{1, ..., N\}\right\}$
        \end{itemize}
        \item Cílem je $F_\theta(x^{(i)}) = y^{(i)} \quad \forall i \in \{1, ..., N\}$  
    \end{itemize}
\end{frame}

% \subsection{Moje počínání}

% \begin{frame}
%     \frametitle{Moje počínání}
%     \begin{itemize}
%         \item Programovací prostředí \emph{pytorch} v jazyce \emph{python}
%         \item Úkol klasifikace číslic z obrázku
%         \begin{itemize}
%             \item Výstupem je potom pravděpodopnostní distribuce
%         \end{itemize}
%         \item Datová sada MNIST \cite{MNIST}
%     \end{itemize}
%     \begin{figure}
%         \includegraphics[scale=0.22]{GRAPHICS/MNIST/output0.png}
%         \includegraphics[scale=0.22]{GRAPHICS/MNIST/output1.png}
%         \includegraphics[scale=0.22]{GRAPHICS/MNIST/output2.png}
%         \includegraphics[scale=0.22]{GRAPHICS/MNIST/output3.png}
%         \includegraphics[scale=0.22]{GRAPHICS/MNIST/output4.png}
%         \includegraphics[scale=0.22]{GRAPHICS/MNIST/output5.png}
%         \centering
%         \caption{Datová sada MNIST}
%     \end{figure}
% \end{frame}

% \begin{frame}
%     \frametitle{Moje počínání}
%     \begin{itemize}
%         \item Ztráta \emph{křížové entropie}
%     \end{itemize}
%     \begin{block}{Ztráta křížové entropie}
%         \centering
%         $L\left(F_\theta(x), y\right) = \sum_{j=1}^n y_j \cdot \operatorname{ln}\left(F_\theta (x)_j\right)$
%     \end{block}
%     \begin{itemize}
%         % \item Algoritmus \emph{RMSProp}
%         \item Úspešnost sítě: 97,57 \%
%     \end{itemize}
% \end{frame}

\section{Adversariální vzorky}

\begin{frame}
    \frametitle{Adversariální vzorek}
    \begin{itemize}
        \item Szegedy a spol. objevili zvláštní chování klasifikačních sítí \cite{szegedy2014intriguing}
    \end{itemize}
    \begin{block}{Existence adversariálních vzorků}
        \centering
        $\exists x, y : \exists \Delta x, \|\Delta x\| \leq \kappa : $

        $C(F(x)) = C(y) \land C(F(x + \Delta x)) \neq C(y)$
    \end{block}
    \begin{itemize}
        \item Označme $\tilde{x} = x + \Delta x$
    \end{itemize}
\end{frame}

\subsection{Metody generování adversariálních vzorků}

\begin{frame}
    \frametitle{Metody generování adversariálních vzorků}
    \begin{itemize}
        \item FGSM
        \begin{itemize}
            \item $\tilde{x} = x + \kappa \cdot \operatorname{sign} \left( \nabla_x L(F(x), y) \right)$
        \end{itemize}
        \item I-FGSM
        \begin{itemize}
            \item $\tilde{x}_0 = x$
            \item $\tilde{x}_{n+1} = \operatorname{Clip}_x^\kappa \{\tilde{x}_n + \gamma \cdot \operatorname{sign}(\nabla_x L(F_{\theta}(x), y))\}$
        \end{itemize}
        \item PGD
        \item Cílená optimalizační metoda
        \begin{itemize}
            \item $\tilde{x} = \operatorname{argmin}_{\hat{x}} \left( \|\hat{x} - x\|
            + \lambda \cdot L(F_{\theta}(\hat{x}), \tilde{y}) \right)$
        \end{itemize}
        \item CW
        \begin{itemize}
            \item $\tilde{x} = \operatorname{argmin}_{\hat{x}} \left(\|\hat{x} - x\|
            - \lambda \cdot L(F_{\theta}(\hat{x}), y) \right)$
        \end{itemize}
    \end{itemize}
    
    % \begin{block}{Fast Gradient Sign Method}
    %     \centering
    %     $\tilde{x} = x + \kappa \cdot \operatorname{sign} \left( \nabla_x L(F_\theta(x), y) \right)$
    % \end{block}

    % \begin{block}{Carlini-Wagner}
    %     \centering
    %     $\tilde{x} = \operatorname{argmin}_{\hat{x}} \left(\|\hat{x} - x\|- c \cdot L(F_{\theta}(\hat{x}), y) \right)$
    % \end{block}
    
\end{frame}

\begin{frame}
    \frametitle{Příklady vzorků vygenerovaných metodami generování adversariálních vzorků}
    % \begin{figure}
    %     \centering
    %     \begin{subfigure}[b]{\textwidth}
    %         \includegraphics[scale=0.22]{GRAPHICS/AEXAMPLES/FGSM/BENIGN/fgsm_benign_0.png}
    %         \includegraphics[scale=0.22]{GRAPHICS/AEXAMPLES/FGSM/BENIGN/fgsm_benign_1.png}
    %         \includegraphics[scale=0.22]{GRAPHICS/AEXAMPLES/FGSM/BENIGN/fgsm_benign_2.png}
    %         \includegraphics[scale=0.22]{GRAPHICS/AEXAMPLES/CW/BENIGN/cw_benign_10.png}
    %         \includegraphics[scale=0.22]{GRAPHICS/AEXAMPLES/CW/BENIGN/cw_benign_5.png}
    %         \includegraphics[scale=0.22]{GRAPHICS/AEXAMPLES/CW/BENIGN/cw_benign_4.png}
    %     \end{subfigure}
    %     \begin{subfigure}[b]{\textwidth}
    %         \includegraphics[scale=0.22]{GRAPHICS/AEXAMPLES/FGSM/ALTERED/fgsm_altered_0.png}
    %         \includegraphics[scale=0.22]{GRAPHICS/AEXAMPLES/FGSM/ALTERED/fgsm_altered_1.png}
    %         \includegraphics[scale=0.22]{GRAPHICS/AEXAMPLES/FGSM/ALTERED/fgsm_altered_2.png}
    %         \includegraphics[scale=0.22]{GRAPHICS/AEXAMPLES/CW/ALTERED/cw_altered_10.png}
    %         \includegraphics[scale=0.22]{GRAPHICS/AEXAMPLES/CW/ALTERED/cw_altered_5.png}
    %         \includegraphics[scale=0.22]{GRAPHICS/AEXAMPLES/CW/ALTERED/cw_altered_4.png}
    %     \end{subfigure}
    %     \caption{FGSM a CW útoky}
    % \end{figure}
    \begin{figure}
        \begin{subfigure}[b]{\textwidth}
            \includegraphics[width=0.09\textwidth]{Images/Graphics/AEXAMPLES/ALLGEN/benign_1.png}
            \includegraphics[width=0.09\textwidth]{Images/Graphics/AEXAMPLES/ALLGEN/benign_3.png}
            \includegraphics[width=0.09\textwidth]{Images/Graphics/AEXAMPLES/ALLGEN/benign_5.png}
            \includegraphics[width=0.09\textwidth]{Images/Graphics/AEXAMPLES/ALLGEN/benign_7.png}
            \includegraphics[width=0.09\textwidth]{Images/Graphics/AEXAMPLES/ALLGEN/benign_2.png}
            \includegraphics[width=0.09\textwidth]{Images/Graphics/AEXAMPLES/ALLGEN/benign_34.png}
            \includegraphics[width=0.09\textwidth]{Images/Graphics/AEXAMPLES/ALLGEN/benign_18.png}
            \includegraphics[width=0.09\textwidth]{Images/Graphics/AEXAMPLES/ALLGEN/benign_37.png}
            \includegraphics[width=0.09\textwidth]{Images/Graphics/AEXAMPLES/ALLGEN/benign_17.png}
            \includegraphics[width=0.09\textwidth]{Images/Graphics/AEXAMPLES/ALLGEN/benign_4.png}
        \end{subfigure}
        \begin{subfigure}[b]{\textwidth}
            \includegraphics[width=0.09\textwidth]{Images/Graphics/AEXAMPLES/ALLGEN/fgsm_1.png}
            \includegraphics[width=0.09\textwidth]{Images/Graphics/AEXAMPLES/ALLGEN/fgsm_3.png}
            \includegraphics[width=0.09\textwidth]{Images/Graphics/AEXAMPLES/ALLGEN/fgsm_5.png}
            \includegraphics[width=0.09\textwidth]{Images/Graphics/AEXAMPLES/ALLGEN/fgsm_7.png}
            \includegraphics[width=0.09\textwidth]{Images/Graphics/AEXAMPLES/ALLGEN/fgsm_2.png}
            \includegraphics[width=0.09\textwidth]{Images/Graphics/AEXAMPLES/ALLGEN/fgsm_34.png}
            \includegraphics[width=0.09\textwidth]{Images/Graphics/AEXAMPLES/ALLGEN/fgsm_18.png}
            \includegraphics[width=0.09\textwidth]{Images/Graphics/AEXAMPLES/ALLGEN/fgsm_37.png}
            \includegraphics[width=0.09\textwidth]{Images/Graphics/AEXAMPLES/ALLGEN/fgsm_17.png}
            \includegraphics[width=0.09\textwidth]{Images/Graphics/AEXAMPLES/ALLGEN/fgsm_4.png}
        \end{subfigure}
        \begin{subfigure}[b]{\textwidth}
            \includegraphics[width=0.09\textwidth]{Images/Graphics/AEXAMPLES/ALLGEN/ifgsm_1.png}
            \includegraphics[width=0.09\textwidth]{Images/Graphics/AEXAMPLES/ALLGEN/ifgsm_3.png}
            \includegraphics[width=0.09\textwidth]{Images/Graphics/AEXAMPLES/ALLGEN/ifgsm_5.png}
            \includegraphics[width=0.09\textwidth]{Images/Graphics/AEXAMPLES/ALLGEN/ifgsm_7.png}
            \includegraphics[width=0.09\textwidth]{Images/Graphics/AEXAMPLES/ALLGEN/ifgsm_2.png}
            \includegraphics[width=0.09\textwidth]{Images/Graphics/AEXAMPLES/ALLGEN/ifgsm_34.png}
            \includegraphics[width=0.09\textwidth]{Images/Graphics/AEXAMPLES/ALLGEN/ifgsm_18.png}
            \includegraphics[width=0.09\textwidth]{Images/Graphics/AEXAMPLES/ALLGEN/ifgsm_37.png}
            \includegraphics[width=0.09\textwidth]{Images/Graphics/AEXAMPLES/ALLGEN/ifgsm_17.png}
            \includegraphics[width=0.09\textwidth]{Images/Graphics/AEXAMPLES/ALLGEN/ifgsm_4.png}
        \end{subfigure}
        \begin{subfigure}[b]{\textwidth}
            \includegraphics[width=0.09\textwidth]{Images/Graphics/AEXAMPLES/ALLGEN/pgd_1.png}
            \includegraphics[width=0.09\textwidth]{Images/Graphics/AEXAMPLES/ALLGEN/pgd_3.png}
            \includegraphics[width=0.09\textwidth]{Images/Graphics/AEXAMPLES/ALLGEN/pgd_5.png}
            \includegraphics[width=0.09\textwidth]{Images/Graphics/AEXAMPLES/ALLGEN/pgd_7.png}
            \includegraphics[width=0.09\textwidth]{Images/Graphics/AEXAMPLES/ALLGEN/pgd_2.png}
            \includegraphics[width=0.09\textwidth]{Images/Graphics/AEXAMPLES/ALLGEN/pgd_34.png}
            \includegraphics[width=0.09\textwidth]{Images/Graphics/AEXAMPLES/ALLGEN/pgd_18.png}
            \includegraphics[width=0.09\textwidth]{Images/Graphics/AEXAMPLES/ALLGEN/pgd_37.png}
            \includegraphics[width=0.09\textwidth]{Images/Graphics/AEXAMPLES/ALLGEN/pgd_17.png}
            \includegraphics[width=0.09\textwidth]{Images/Graphics/AEXAMPLES/ALLGEN/pgd_4.png}
        \end{subfigure}
        \begin{subfigure}[b]{\textwidth}
            \includegraphics[width=0.09\textwidth]{Images/Graphics/AEXAMPLES/ALLGEN/lbfgs_1.png}
            \includegraphics[width=0.09\textwidth]{Images/Graphics/AEXAMPLES/ALLGEN/lbfgs_3.png}
            \includegraphics[width=0.09\textwidth]{Images/Graphics/AEXAMPLES/ALLGEN/lbfgs_5.png}
            \includegraphics[width=0.09\textwidth]{Images/Graphics/AEXAMPLES/ALLGEN/lbfgs_7.png}
            \includegraphics[width=0.09\textwidth]{Images/Graphics/AEXAMPLES/ALLGEN/lbfgs_2.png}
            \includegraphics[width=0.09\textwidth]{Images/Graphics/AEXAMPLES/ALLGEN/lbfgs_34.png}
            \includegraphics[width=0.09\textwidth]{Images/Graphics/AEXAMPLES/ALLGEN/lbfgs_18.png}
            \includegraphics[width=0.09\textwidth]{Images/Graphics/AEXAMPLES/ALLGEN/lbfgs_37.png}
            \includegraphics[width=0.09\textwidth]{Images/Graphics/AEXAMPLES/ALLGEN/lbfgs_17.png}
            \includegraphics[width=0.09\textwidth]{Images/Graphics/AEXAMPLES/ALLGEN/lbfgs_4.png}
        \end{subfigure}
        \begin{subfigure}[b]{\textwidth}
            \includegraphics[width=0.09\textwidth]{Images/Graphics/AEXAMPLES/ALLGEN/cw_1.png}
            \includegraphics[width=0.09\textwidth]{Images/Graphics/AEXAMPLES/ALLGEN/cw_3.png}
            \includegraphics[width=0.09\textwidth]{Images/Graphics/AEXAMPLES/ALLGEN/cw_5.png}
            \includegraphics[width=0.09\textwidth]{Images/Graphics/AEXAMPLES/ALLGEN/cw_7.png}
            \includegraphics[width=0.09\textwidth]{Images/Graphics/AEXAMPLES/ALLGEN/cw_2.png}
            \includegraphics[width=0.09\textwidth]{Images/Graphics/AEXAMPLES/ALLGEN/cw_34.png}
            \includegraphics[width=0.09\textwidth]{Images/Graphics/AEXAMPLES/ALLGEN/cw_18.png}
            \includegraphics[width=0.09\textwidth]{Images/Graphics/AEXAMPLES/ALLGEN/cw_37.png}
            \includegraphics[width=0.09\textwidth]{Images/Graphics/AEXAMPLES/ALLGEN/cw_17.png}
            \includegraphics[width=0.09\textwidth]{Images/Graphics/AEXAMPLES/ALLGEN/cw_4.png}
        \end{subfigure}
        \centering
        \caption{Vzorky generované různými metodami hledání adversariálních vzorků}
        \label{allgen}
    \end{figure}
\end{frame}

\section{Robustní učení}

\begin{frame}
    \frametitle{Robustní učení}
    \begin{itemize}
        \item Snaha o \emph{robustnost} klasifikátorů proti adversariálním útokům
    \end{itemize}

    \begin{block}{Obecná formulace problému}
        \centering
        $\theta = \operatorname{argmin}_\theta \frac{1}{N} \sum_{i=1}^N  \operatorname{max}_{\tilde{x} \in B(x^{(i)}, \kappa)} L\left(F_\theta(\tilde{x}), y^{(i)}\right)$
    \end{block}
\end{frame}

% \subsection{Adversariální trénink}

% \begin{frame}
%     \frametitle{Adversariální trénink}
%     \begin{itemize}
%         \item Přidání adversariálních vzorků do trénovací sady
%         \item Zvyšuje robustnost sítě \cite{GoodfellowEtAl}
%         \item Avšak požaduje vyšší kapacitu
%     \end{itemize}
% \end{frame}

\subsection{Úspěšnost metod generování adversariálních vzorků}

\begin{frame}
    \frametitle{Úspěšnost metod generování adversariálních vzorků}
    \begin{table}
        \centering
        \begin{tabular}{| c | c | c |}
            \hline
            Metoda & Úspěšnost & Robustní úspěšnost\\
            \hline
            FGSM & 40.4 \% & 5.7 \% \\
            I-FGSM & 78.4 \% & 7.0 \% \\
            PGD & 78.8 \% & 6.9 \% \\
            Cílená optimalizační metoda & 100 \% & 100 \% \\
            CW & 99.7 \% & 100 \% \\
            \hline
        \end{tabular}
        \caption{Úspěšnost metod generování adversariálních vzorků}
        \label{allgen_success}
    \end{table}
\end{frame}

\begin{frame}
    \frametitle{Příklady vzorků vygenerovaných metodami generování adversariálních vzorků proti robustně naučené síti}
    \begin{figure}
        \begin{subfigure}[b]{\textwidth}
            \includegraphics[width=0.09\textwidth]{Images/Graphics/AEXAMPLES/R_ALLGEN/benign_1.png}
            \includegraphics[width=0.09\textwidth]{Images/Graphics/AEXAMPLES/R_ALLGEN/benign_3.png}
            \includegraphics[width=0.09\textwidth]{Images/Graphics/AEXAMPLES/R_ALLGEN/benign_5.png}
            \includegraphics[width=0.09\textwidth]{Images/Graphics/AEXAMPLES/R_ALLGEN/benign_7.png}
            \includegraphics[width=0.09\textwidth]{Images/Graphics/AEXAMPLES/R_ALLGEN/benign_2.png}
            \includegraphics[width=0.09\textwidth]{Images/Graphics/AEXAMPLES/R_ALLGEN/benign_34.png}
            \includegraphics[width=0.09\textwidth]{Images/Graphics/AEXAMPLES/R_ALLGEN/benign_18.png}
            \includegraphics[width=0.09\textwidth]{Images/Graphics/AEXAMPLES/R_ALLGEN/benign_37.png}
            \includegraphics[width=0.09\textwidth]{Images/Graphics/AEXAMPLES/R_ALLGEN/benign_17.png}
            \includegraphics[width=0.09\textwidth]{Images/Graphics/AEXAMPLES/R_ALLGEN/benign_4.png}
        \end{subfigure}
        \begin{subfigure}[b]{\textwidth}
            \includegraphics[width=0.09\textwidth]{Images/Graphics/AEXAMPLES/R_ALLGEN/fgsm_1.png}
            \includegraphics[width=0.09\textwidth]{Images/Graphics/AEXAMPLES/R_ALLGEN/fgsm_3.png}
            \includegraphics[width=0.09\textwidth]{Images/Graphics/AEXAMPLES/R_ALLGEN/fgsm_5.png}
            \includegraphics[width=0.09\textwidth]{Images/Graphics/AEXAMPLES/R_ALLGEN/fgsm_7.png}
            \includegraphics[width=0.09\textwidth]{Images/Graphics/AEXAMPLES/R_ALLGEN/fgsm_2.png}
            \includegraphics[width=0.09\textwidth]{Images/Graphics/AEXAMPLES/R_ALLGEN/fgsm_34.png}
            \includegraphics[width=0.09\textwidth]{Images/Graphics/AEXAMPLES/R_ALLGEN/fgsm_18.png}
            \includegraphics[width=0.09\textwidth]{Images/Graphics/AEXAMPLES/R_ALLGEN/fgsm_37.png}
            \includegraphics[width=0.09\textwidth]{Images/Graphics/AEXAMPLES/R_ALLGEN/fgsm_17.png}
            \includegraphics[width=0.09\textwidth]{Images/Graphics/AEXAMPLES/R_ALLGEN/fgsm_4.png}
        \end{subfigure}
        \begin{subfigure}[b]{\textwidth}
            \includegraphics[width=0.09\textwidth]{Images/Graphics/AEXAMPLES/R_ALLGEN/ifgsm_1.png}
            \includegraphics[width=0.09\textwidth]{Images/Graphics/AEXAMPLES/R_ALLGEN/ifgsm_3.png}
            \includegraphics[width=0.09\textwidth]{Images/Graphics/AEXAMPLES/R_ALLGEN/ifgsm_5.png}
            \includegraphics[width=0.09\textwidth]{Images/Graphics/AEXAMPLES/R_ALLGEN/ifgsm_7.png}
            \includegraphics[width=0.09\textwidth]{Images/Graphics/AEXAMPLES/R_ALLGEN/ifgsm_2.png}
            \includegraphics[width=0.09\textwidth]{Images/Graphics/AEXAMPLES/R_ALLGEN/ifgsm_34.png}
            \includegraphics[width=0.09\textwidth]{Images/Graphics/AEXAMPLES/R_ALLGEN/ifgsm_18.png}
            \includegraphics[width=0.09\textwidth]{Images/Graphics/AEXAMPLES/R_ALLGEN/ifgsm_37.png}
            \includegraphics[width=0.09\textwidth]{Images/Graphics/AEXAMPLES/R_ALLGEN/ifgsm_17.png}
            \includegraphics[width=0.09\textwidth]{Images/Graphics/AEXAMPLES/R_ALLGEN/ifgsm_4.png}
        \end{subfigure}
        \begin{subfigure}[b]{\textwidth}
            \includegraphics[width=0.09\textwidth]{Images/Graphics/AEXAMPLES/R_ALLGEN/pgd_1.png}
            \includegraphics[width=0.09\textwidth]{Images/Graphics/AEXAMPLES/R_ALLGEN/pgd_3.png}
            \includegraphics[width=0.09\textwidth]{Images/Graphics/AEXAMPLES/R_ALLGEN/pgd_5.png}
            \includegraphics[width=0.09\textwidth]{Images/Graphics/AEXAMPLES/R_ALLGEN/pgd_7.png}
            \includegraphics[width=0.09\textwidth]{Images/Graphics/AEXAMPLES/R_ALLGEN/pgd_2.png}
            \includegraphics[width=0.09\textwidth]{Images/Graphics/AEXAMPLES/R_ALLGEN/pgd_34.png}
            \includegraphics[width=0.09\textwidth]{Images/Graphics/AEXAMPLES/R_ALLGEN/pgd_18.png}
            \includegraphics[width=0.09\textwidth]{Images/Graphics/AEXAMPLES/R_ALLGEN/pgd_37.png}
            \includegraphics[width=0.09\textwidth]{Images/Graphics/AEXAMPLES/R_ALLGEN/pgd_17.png}
            \includegraphics[width=0.09\textwidth]{Images/Graphics/AEXAMPLES/R_ALLGEN/pgd_4.png}
        \end{subfigure}
        \begin{subfigure}[b]{\textwidth}
            \includegraphics[width=0.09\textwidth]{Images/Graphics/AEXAMPLES/R_ALLGEN/lbfgs_1.png}
            \includegraphics[width=0.09\textwidth]{Images/Graphics/AEXAMPLES/R_ALLGEN/lbfgs_3.png}
            \includegraphics[width=0.09\textwidth]{Images/Graphics/AEXAMPLES/R_ALLGEN/lbfgs_5.png}
            \includegraphics[width=0.09\textwidth]{Images/Graphics/AEXAMPLES/R_ALLGEN/lbfgs_7.png}
            \includegraphics[width=0.09\textwidth]{Images/Graphics/AEXAMPLES/R_ALLGEN/lbfgs_2.png}
            \includegraphics[width=0.09\textwidth]{Images/Graphics/AEXAMPLES/R_ALLGEN/lbfgs_34.png}
            \includegraphics[width=0.09\textwidth]{Images/Graphics/AEXAMPLES/R_ALLGEN/lbfgs_18.png}
            \includegraphics[width=0.09\textwidth]{Images/Graphics/AEXAMPLES/R_ALLGEN/lbfgs_37.png}
            \includegraphics[width=0.09\textwidth]{Images/Graphics/AEXAMPLES/R_ALLGEN/lbfgs_17.png}
            \includegraphics[width=0.09\textwidth]{Images/Graphics/AEXAMPLES/R_ALLGEN/lbfgs_4.png}
        \end{subfigure}
        \begin{subfigure}[b]{\textwidth}
            \includegraphics[width=0.09\textwidth]{Images/Graphics/AEXAMPLES/R_ALLGEN/cw_1.png}
            \includegraphics[width=0.09\textwidth]{Images/Graphics/AEXAMPLES/R_ALLGEN/cw_3.png}
            \includegraphics[width=0.09\textwidth]{Images/Graphics/AEXAMPLES/R_ALLGEN/cw_5.png}
            \includegraphics[width=0.09\textwidth]{Images/Graphics/AEXAMPLES/R_ALLGEN/cw_7.png}
            \includegraphics[width=0.09\textwidth]{Images/Graphics/AEXAMPLES/R_ALLGEN/cw_2.png}
            \includegraphics[width=0.09\textwidth]{Images/Graphics/AEXAMPLES/R_ALLGEN/cw_34.png}
            \includegraphics[width=0.09\textwidth]{Images/Graphics/AEXAMPLES/R_ALLGEN/cw_18.png}
            \includegraphics[width=0.09\textwidth]{Images/Graphics/AEXAMPLES/R_ALLGEN/cw_37.png}
            \includegraphics[width=0.09\textwidth]{Images/Graphics/AEXAMPLES/R_ALLGEN/cw_17.png}
            \includegraphics[width=0.09\textwidth]{Images/Graphics/AEXAMPLES/R_ALLGEN/cw_4.png}
        \end{subfigure}
        \centering
        \caption{Vzorky generované metodami hledání adversariálních vzorků proti robustně naučené síti}
        \label{r_allgen}
    \end{figure}
\end{frame}

\section*{Závěr}

\begin{frame}
    \frametitle{Závěr}
    \begin{itemize}
        \item Metody strojového učení skrývají úskalí
        \item Lze snadno zneužít adversariálních vzorků
        \item Proti takovým útokům se však lze bránit
    \end{itemize}
\end{frame}

\section*{Literatura}

\begin{frame}
    \frametitle{Literatura}
    \begin{thebibliography}{1}
        \bibitem{Goodfellow}I. Goodfellow, Y. Bengio, A. Courville,
        \emph{Deep Learning}. MIT Press, 2016.

        % \bibitem{MNIST} Y. Lecun, C. Cortes, C. J. Burges,
        % \emph{The mnist database of handwritten digits}. 1998.

        \bibitem{szegedy2014intriguing} C. Szegedy, W. Zaremba, I. Sutskever, J. Bruna, D. Erhan, I. Goodfellow, R. Fergus,
        \emph{Intriguing properties of neural networks}.
        arXiv, 2014.

        % \bibitem{GoodfellowEtAl} I. Goodfellow, J. Shlens, C. Szegedy,
        % \emph{Explaining and Harnessing Adversarial Examples}.
        % In 'International Conference on Learning Representations', ICLR 2015.
    \end{thebibliography}
\end{frame}

\section{Otázky}

\begin{frame}
    \frametitle{Hledání $\lambda$ pomocí bisekce}
    \begin{itemize}
        \item Připomenutí: $\tilde{x}  = \operatorname{argmin}_{\hat{x}} \left(\|\hat{x} - x\|
        - \lambda \cdot L(F_{\theta}(\hat{x}), y) \right)$
        \item Cíl: Nalézt $\lambda$, pro které je řešení problému nesprávně klasifikováno a zároveň $\|\tilde{x} - x\|$ je co nejmenší
        \item Definuji pomocnou funkci $g: (0, + \infty) \rightarrow \{-1, 1\}$
        \begin{itemize}
            \item $g(\lambda) = 1$, pokud $\tilde{x}$ pro ono $\lambda$ je nesprávně klasifikováno
            \item $g(\lambda) = -1$, pokud $\tilde{x}$ pro ono $\lambda$ je správně klasifikováno
        \end{itemize}
    \end{itemize}
\end{frame}

\begin{frame}
    \frametitle{Hledání $\lambda$ pomocí bisekce}
    \begin{enumerate}
        \item Nalézt $\nu$, pro které $g(\nu) = 1$
        \item Nalézt $\mu$, pro které $g(\mu) = -1$
        \item Ozn. $\lambda = \mu + \frac{\nu - \mu}{2}$
        \begin{itemize}
            \item Pro $g(\lambda) = 1$ provést $\nu \leftarrow \lambda$
            \item Pro $g(\lambda) = -1$ provést $\mu \leftarrow \lambda$
        \end{itemize}
        \item Opakovat až $\nu - \mu < \varepsilon$
        \item Vrátit $\nu$
    \end{enumerate}
\end{frame}

\begin{frame}
    \frametitle{Bisekce a konvergence}
    \begin{itemize}
        \item Věta o konvergenci bisekce požaduje spojitost funkce, jejíž kořeny se hledají
        \item To ovšem $g$ není
        \item Nicméně posloupnost čísel $\nu$ je Cauchyovská v $\mathbb{R}$
        \item Čili, lze-li provést každý krok algoritmu, pak metoda konverguje
    \end{itemize}
\end{frame}

\begin{frame}
    \frametitle{Cílené FGSM}
    \begin{itemize}
        \item Předpis cílené FGSM: $\tilde{x} = x - \kappa \cdot \operatorname{sign} \left( \nabla_x L(F(x), \tilde{y}) \right)$
    \end{itemize}
\end{frame}

\begin{frame}
    \frametitle{Počet iterací tréningu neuronové sítě}
    \begin{itemize}
        \item Volba probíhala s přihlédnutím k následujícím faktorům:
        \begin{itemize}
            \item Dostatečný počet vzorků, které síť během trénování potká
            \begin{itemize}
                \item $5000 \cdot 30 = 150000 > 60000$
            \end{itemize}
            \item Úspěšnost sítě na testovací datové sadě
            \item CPU čas (trénoval jsem na CPU)
        \end{itemize}
    \end{itemize}
\end{frame}

\begin{frame}
    \frametitle{Porovnávání úspěšnosti útoku}
    \begin{itemize}
        \item Čistě porovnání procentuální úspěšnosti útoku nezohledňuje blízkost adversariálního vzorku k benignímu
        \item Čili jako srovnání metod by byla lepší dvojice čísel, a to procentuální úspěšnost útoku a statistika norem perturbací (např. průměr)
    \end{itemize}
\end{frame}

\begin{frame}
    \frametitle{Volba použité normy}
    \begin{itemize}
        \item Blíže bude vzorek s euklidovskou normou perturbace menší než $0.5$ oproti vzorku s maximovou normou perturbace menší než $0.5$
    \end{itemize}
\end{frame}

\end{document}